\documentclass{article}
\usepackage[utf8]{inputenc}
\usepackage{amssymb}
\usepackage{amsmath}
\usepackage{amsthm}
\usepackage{siunitx}
\usepackage{float}
\usepackage{hyperref}

\newtheorem*{remark}{Remark}
\newtheorem*{idea}{Idea}

\title{3BOX1 - List of formulas}
\author{Xander Smeets}
\date{December 2019}

\begin{document}
\maketitle

\section{Wave equation}
A function describes a travelling wave if it adheres to the wave equation.
It will do so if the function is a function of $x+vt$ or $x-vt$.
A common way to write a wave is using $$\psi{}=A e^{i\left(k\cdot r + \omega t + \phi\right)}$$

$$B_0=\frac{E_0}{c}$$
\begin{idea}
An easy way to remember the order of the symbols in this formula is by considering the fact that the electric field $E_0$ is always much larger than the magnetic field $B_0$.
\end{idea}

\noindent{}Radiation pressure for $100\%$ absorption: $P=\frac{I}{c}$.\newline
Radiation pressure for $100\%$ reflection: $P=\frac{2I}{c}$.

$$\epsilon_0=\num{8.854e-12}\;  \si{\farad\per\metre}$$

\section{Reflection and Refraction}

Critical angle (for total internal reflection): $\sin(\theta{}_c)=\frac{n_t}{n_i}$\newline
\begin{remark}
This is similar to Brewster's angle, but with a sine instead of a tangent.
\end{remark}

\subsection{Fresnel equations}
The Fresnel equations give a relation between the reflectivity $R$, the indices of refraction, the angle of incidence and the angle of transmission.
\begin{remark}
Watch out!
The equations given on the formula sheet give the reflection ratio for the \emph{electric field} $r$, while the often-needed reflectivity $R$ is the ratio between the incident and reflected \emph{power}.
From the relation between electric field and power, it then follows that $R=r^2$, hence the relations on the sheet have to be squared (most of the time).
\end{remark}
\begin{remark}
For the transmissivity $T$, it works differently once again.
$$T=\frac{P_t}{P_i}=\frac{I_t\cos(\theta_t)}{I_i\cos(\theta_i)}=\frac{n_t\cos(\theta_t)}{n_i\cos(\theta_i)}t^2$$
\end{remark}

$$\omega=2\pi f=\frac{2\pi c}{\lambda}$$

\noindent
For reflectivity:
\begin{equation}
    R=\frac{P_r}{P_i}=\frac{I_r}{I_i}=\frac{E_{0r}^2}{E_{0i}^2}=r^2
\end{equation}
For transmissivity:
\begin{equation}
    T=\frac{P_t}{P_i}=\frac{I_t \cos{\theta_t}}{I_i \cos{\theta_i}}=\left(\frac{n_t\cos{\theta_t}}{n_i\cos{\theta_i}}\right)t^2=n\left(\frac{\cos{\theta_t}}{\cos{\theta_i}}\right)t^2
\end{equation}

\section{Geometric optics 1}

\begin{table}[H]
    \centering
    \begin{tabular}{c|c}
         convex & bol \\ \hline
         concave & hol
    \end{tabular}
    \caption{Convex and concave}
    \label{tab:my_label}
\end{table}
% This is a bit hard to remember; think of the word convex, which contains the letter x. The letter x looks similar to a concave lens (so it is just the other way around :) )

Construction rays for a concave lens:
\begin{enumerate}
    \item A ray which goes straight (parallel to the optical axis) towards the lens and then `reflects' to go through the focal point.
    \item A ray through the middle of the lens.
\end{enumerate}

\section{Geometric optics 2}
Concave mirror: light is `in the circle'; $R>0$\newline
Convex mirror: light is outside of the circle (think of the word `\emph{exit}'); $R<0$

\noindent{}Magnification in general: $M=-\frac{s_i}{s_o}$
\begin{remark}
For the magnification: watch out for the minus sign!
\end{remark}

\noindent{}Maximum angular magnification of a magnifier, as well as the magnification of the ocular of a microscope:
$M_A=M_e=\frac{N.P.}{f}$, where $N.P.=254$ \si{\milli\metre}, the near point distance of the eye.

\subsection{System matrices}

\noindent{}For a displacement $d$:
$
\begin{bmatrix}
1 & d\\
0 & 1
\end{bmatrix}
$

\noindent{}For a thin lens with focal length $f$:
$
\begin{bmatrix}
1 & 0\\
-\frac{1}{f} & 1
\end{bmatrix}
$

\noindent{}For refraction on a spherical surface with curvature $R$:
$
\begin{bmatrix}
1 & 0\\
-\frac{n_t-n_i}{n_t}\frac{1}{R} & \frac{n_i}{n_t}
\end{bmatrix}
$

\noindent{}For reflection on a mirror with curvature $R$:
$
\begin{bmatrix}
1 & 0\\
\frac{2}{R} & 1
\end{bmatrix}
$

\begin{remark}
For a planar surface, note that the radius of curvature is infinite; $R=\infty$
\end{remark}

\section{Superposition and Polarization}
$$v_p=\frac{\omega}{k}=\frac{c}{n}=f\lambda$$
$$k=\frac{2\pi}{\lambda}$$

\noindent{}When `completely describing' the state of polarization of a wave, one can often follow these steps:
\begin{enumerate}
\item Calculate and simplify $E_x$ and $E_y$, usually at $z=0$ (so the $kz$-term drops out).
\item Determine the phase difference between $E_x$ and $E_y$.
\item Determine whether the wave is polarised
\begin{itemize}
\item linearly (at $-\pi$, $0$ or $\pi$ radians)
\item circularly (right-cicular at $-\frac{\pi}{2}$, left-circular at $\frac{\pi}{2}$ radians)
\item elliptically (right-elliptically for negative phase differences, left-elliptically for positive phase differences)
\end{itemize}
\end{enumerate}
\section{Polarization}
\noindent{}In a Michelson interferometer, as the two beams interfere constructively whenever the optical paths differ an integer number of wavelengths: $$OPD=2\Delta L = m \lambda_0$$
where $m$ is the number of fringe pairs observed when moving the mirror over a distance $\Delta L$.
\newline\newline
Given a Jones vector $J$, we know that
\begin{equation}
    I=\frac{1}{2} c \epsilon_0 \left|J\right|^2
\end{equation}
\iffalse
\noindent
A wave plate is defined to change the polarization state of a wave passing through it.
This means that the definition of a half-wave plate, for example, is that it changes the polarization angle of the light along the slow axis by $\pi$.
The definition \textbf{does not} state that a delay of $\frac{1}{2}\lambda$ is induced along the slow axis (although this is usually the case (especially, the plate does not induce a delay of $\frac{1}{2}\lambda_0$).
\fi
\section{Interference}
\begin{idea}
The reflection phase change is determined according to the following rule:
``When the index of refraction of the incident medium is less than that of the transmitted medium, the reflected beam is phase-shifted by $\pi$ radians.'' (Bennett, page 207)
\end{idea}
\noindent{}A somewhat shorter way of remembering this idea:
$$n_i<n_t\implies \Delta\phi_{ref} \; + \! = \; \pi$$
\begin{idea}
Maybe even: the ILT (government authority) wants change (incident less than transmitted gives change of the angle).
Then the trick is $I$ Less than $T$.
\end{idea}
\section{Diffraction}
\subsection{Resolving power}
Resolving power is the ability of an imaging device to separate (to see as distinct) points of an object that are located at a small angular distance or it is the power of an optical instrument to separate far away objects, that are close together, into individual images.\footnote{Wikipedia: \url{https://en.wikipedia.org/wiki/Angular_resolution}}
$$\mathcal{R}=\frac{\lambda}{\Delta\lambda}=Nm$$
\subsection{Rayleigh criterion}
The Rayleigh criterion for barely resolving two objects that are point sources of light, such as stars seen through a telescope, is that the center of the Airy disk for the first object occurs at the first minimum of the Airy disk of the second.\footnote{Wikipedia: \url{https://en.wikipedia.org/wiki/Airy_disk}}
$$\theta=1.22 \frac{\lambda}{d}$$
\subsection{Missing orders}
A missing order occurs when $\frac{d}{b}\in{\mathbb{N}}$.
If this is the case, then every $\frac{d}{b}^{\text{th}}$ order will be missing.
\iffalse
The angle for which this occurs can be calculated by using $$\gamma_{n\cdot\frac{d}{b}}=\left(n\cdot\frac{d}{b}\right)\pi=\frac{1}{2}k d \sin\left({\theta_{n\cdot\frac{d}{b}}}\right)=\frac{\pi}{\lambda_0}n d \sin\left(\theta_{n\cdot\frac{d}{b}}\right)$$
\fi
\end{document}
